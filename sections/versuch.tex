\section{Der Versuch}
\subsection{Anodische Oxidation von Aluminium}
\begin{longtable}{p{3cm}p{14cm}}
	\textbf{Vorbereitung}
		& 
		\begin{enumerate}
			\item Zwei Gemüseschäler mit Aluminiumdraht jeweils an einem Elektrodenbügel aufhängen und mit einer Klammer fixieren. Der Draht soll durch alle verfügbaren Ösen geführt werden.
			
			\item Am Elektrodenbügel und am Schäler auf guten Kontakt achten.
			
			\item Ein Referenzaluplättchen mittels Klammer am Elektrodenbügel befestigen. Dies dient zur späteren Bestimmung der Schichtdicke.
		\end{enumerate}\\
	\hline
	\textbf{Beizen}
		& 
		\begin{enumerate}
			\item Proben in eine 15\% \chemfig{NaOH}-Lösung tauchen.
			
			\item Eine starke Blasenbildung setzt an den Proben ein.
			
			\item Nach 5 Minuten kann der Vorgang beendet werden.
			
			\item Die Proben ca. 5 Minuten im Spülbecken spülen.
		\end{enumerate}\\
	\textbf{Eloxieren}
		& 
		\begin{enumerate}
			\item Auf guten Kontakt der Aluelektroden achten
			
			\item Spannung auf 15V einstellen
			
			\item Die Proben in eine 20\% \chemfig{H_{2}SO_{4}}-Lösung tauchen.
			
			\item Die Probem währen 1 Stunde im Bad bewegen.
			
			\item In regelmässigen Abständen der Kontakt der Elektroden überprüfen (gleichmässige Blasenbildung).
			
			\item Die Werkstücke während 5 Minuten in der Spülung wässern.
		\end{enumerate}\\
	\hline
	\textbf{Bestimmung der Schichtdicke}
		&
		\begin{enumerate}
			\item Referenzaluplättchen in den Trockenschrank geben.
			
			\item Das Plättchen auf der Analysewaage wägen.
			
			\item Das Plättchen für 15 Minuten in kochende Chrom-Phosphorsäure-Lösung geben.
			
			\item Das Plättchen gut spülen.
			
			\item Zur Überprüfung das Plättchen für ca. 1 Minute in ein Farbbad geben, das Plättchen soll keine Farbe annehmen.
			
			\item Das Plättchen erneut spülen und trocknen.
			
			\item Das Plättchen erneut wägen. Die Massendifferenz entspricht der Masse der entfernten Eloxalschicht.
		\end{enumerate}
		
		$$\boxed{
			\begin{aligned}
			m_{Eloxal} 		&= \text{Masse der entfernten Eloxalschicht}\\
			\rho_{Eloxal} 	&= 	\begin{cases}
									2.5\cdot 10^{6} \frac{\mathrm{g}}{\mathrm{m}^3} \quad \text{im unverdichteten Zustand}\\
									2.7\cdot 10^{6} \frac{\mathrm{g}}{\mathrm{m}^3} \quad \text{im verdichteten Zustand}\\
 								\end{cases}\\
 				V_{Eloxal}	&= \frac{m_{Eloxal}}{\rho_{Eloxal}}\\
 				d_{Eloxal} 	&\approx \frac{V_{Eloxal}}{2\cdot h \cdot b}
 			\end{aligned}}$$\\
 	\hline
 	\textbf{Einfärben der Kartoffelschäler}
 		&
 			\begin{enumerate}
 				\item Die Schäler während 5 Minuten in das Farbbad geben.
 				
 				\item Die Schäler während 5 Minuten spülen.
 			\end{enumerate}\\
 	\hline
 	\textbf{Verdichten}
 		& 
 			\begin{enumerate}
 				\item Die eingefärbten Teile während einer $\frac{3}{4}$ Stunde in einen Alutopf mit kochendem deionisierten Wasser geben.
 			\end{enumerate}\\
 	\hline
\end{longtable}

\newpage

\subsection{Galvanisches Verkupfern, Vernickeln und Vergolden}
\begin{longtable}{p{3cm}p{14cm}}
	\textbf{Grobreinigung}
		& 
		\begin{enumerate}
			\item Die Proben mit warmem Wasser Anfeuchten.
			
			\item Durch kräftiges Reiben mit Scheuermittel von Hand die Probenoberfläche reinigen.
			
			\item Die Proben für ca. 5 Minuten in ein Ultraschallbad mit lauwarmem Wasser und Seife reinigen und spülen.
		\end{enumerate}\\
	\hline
	\textbf{Elektrolytisches Entfetten}
		& 
		\begin{enumerate}
			\item Die Proben an Probehaltern aus \chemfig{Cu}-Draht am Elektrodenbügel befestigen.
			
			\item Die Proben bei ca 40$^\circ$C für 30 Sekunden ins Entfettungsbad geben. $\mathrm{I}_{Max}=8\mathrm{A}$. Maximal mögliche Spannung einstellen.
			
			\item Die Proben während ca 30 Sekunden gut wässern.
		\end{enumerate}\\
	\hline
	\textbf{Dekapieren}
		& 
		\begin{enumerate}
			\item Die Proben für ca. 10 Minuten ins Dekapierbad geben.
			
			\item Die Proben ca. 20 Sekunden wässern.
		\end{enumerate}\\
	\hline
	\textbf{Verkupfern}
		& 
		\textcolor{red}{\textbf{Achtung: Die Elektrolytlösung darf unter keinen Umständen sauer werden!! Sonst entsteht aus dem Zyanid hochtoxische Blausäure, welche als Gas aus der Lösung entweicht!!}} 
		$$\chemfig{CN^{-}} + \chemfig{H_{3}O^{+}} \chemrel{->} \chemfig{HCN} + \chemfig {H_{2}O}$$
		
		\begin{enumerate}
			\item Spannung von ca 4V anlegen.
			
			\item Strom einstellen: 	$$\begin{aligned}
											J &= 100 \frac{\mathrm{A}}{\mathrm{m}^2}\\
											I &= J\cdot A
										\end{aligned}$$
			
			\item Proben für 10 Sekunden bei Raumtemperatur ohne Bewegung in das Bad geben.
			
			\item Die Proben zwischenspülen und während 20 Sekunden gut wässern.
		\end{enumerate}\\
	\hline
	\textbf{Vernickeln}
		&
		\begin{enumerate}
			\item Spannung von 1.9V anlegen, \underline{höchstens} 2V!
			
			\item Strom einstellen: 	$$\begin{aligned}
											J &= 500 \frac{\mathrm{A}}{\mathrm{m}^2}\\
											I &= J\cdot A
										\end{aligned}$$
										
			\item Die Proben für ca. 8 Minuten in das 50$^\circ$C warme Bad geben. Die Proben müssen dabei bewegt werden.
			
			\item Die Proben zwischenspülen und während 20 Sekunden gut wässern.
		\end{enumerate}\\
	\hline
	\textbf{Vergolden}
		&
		\begin{enumerate}
			\item Spannung von 2.9V anlegen.
			
			\item Strom einstellen: 	$$\begin{aligned}
											J &= 45 \frac{\mathrm{A}}{\mathrm{m}^2}\\
											I &= J\cdot A
										\end{aligned}$$
										
			\item Die Proben für ca. 7 Minuten bei Raumtemperatur in das Bad geben. Die Proben müssen dabei bewegt werden.
			
			\item Die Proben zwischenspülen und während 1 Minute gut wässern.
			
			\item Die Proben abtrocknen und mit Glanzmittel auf Hochglanz polieren.
			
			\item Berechnung der abgeschiedenen Goldmenge:
			$$ \boxed{
				\begin{aligned}
					z &= \text{Anzahl verschobener Elektronen pro Einheit} = 1\\
					F &= \text{Faraday-Konstante} = 96'500 \frac{\mathrm{C}}{\mathrm{mol}}\\
					r &= \text{Wirkungsgrad/Stromausbeute}\\
					n &= \text{Stoffmenge}\\
					t &= \text{Zeit}\\
					M &= \text{Molare Masse von Gold} = 196.97 \cdot 10^{-3} \frac{\mathrm{kg}}{\mathrm{mol}}\\
					Q &= I \cdot t = n\cdot z \cdot F\\
					n &= \frac{m}{M}\\
					\Longleftrightarrow m &= \frac{M \cdot I \cdot t}{z\cdot F} \cdot r\\
				\end{aligned}}$$
														
			\item Berechnung der Dicke der Goldschicht:
			$$ \boxed{
				\begin{aligned}
					m 	&= \rho \cdot V = \rho \cdot A \cdot d\\
					\Longleftrightarrow d	&= \frac{m}{\rho \cdot A}
				\end{aligned}}$$
			\item Berechnung des Goldwerts $x$ bei einem Goldpreis von $p_x = 25'000 \frac{\mathrm{Fr.}}{\mathrm{kg}}$: $$\boxed{x = m \cdot  p_x}$$
		\end{enumerate}
\end{longtable}

\newpage