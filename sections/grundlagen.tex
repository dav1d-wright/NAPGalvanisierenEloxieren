\section{Grundlagen}

\subsection{Wertigkeit}
\begin{longtable}{p{3cm}p{14cm}}
	\textbf{Definition}
		& Die Wertigkeit ist in der Chemie kein einheitlich benutzter Begriff. Meist wird unter der Wertigkeit, oder Valenz, eines Atoms verstanden, welche Anzahl Elektronen für eine Verbindung zur Verfügung steht. Dem entsprechend ist sie keine elementspezifische Grösse und ein Stoff kann gemäss der jeweiligen Bindungssituation in unterschiedlichen Wertigkeiten vorkommen. Die Summe der Wertigkeiten in Verbindungen ist immer 0. Eine positive Wertigkeit zeigt an, dass die Elektronendichte gegenüber seinem Normalzustand verringert ist, eine negative zeigt an, dass die Elektronendichte um das Atom erhöht ist. Abgesehen von der Wertigkeit 0 wird sie ausschliesslich in römischen Zahlen angegeben.\\
\end{longtable}

\subsection{Redoxreaktionen}
\begin{longtable}{p{3cm}p{14cm}}
	\textbf{Definition}
		& Redoxreaktionen sind Reaktionen, in denen gleichzeitig ein Stoff reduziert und ein anderer oxidiert wird. Daher stammt ihr ursprünglicher Name: \textbf{Reduktions-Oxidations-Reaktion}. Redoxreaktionen sind in der Chemie von grundlegender Bedeutung. Vor allem Verbrennungs- und Stoffwechselvorgägne sowie viele technische Produktionsprozesse basieren auf diesen Prozessen.\\
	\hline
	\textbf{Reduktion}
		& Bei einer Reduktion nimmt die Elektronendichte (Elektronegativität) um ein Atom zu, dieser Stoff wird \textbf{Oxidationsmittel} oder \textbf{Akzeptor} genannt. Es bekommt sozusagen vom Reaktionspartner ein Elektron zur Verfügung gestellt. Dabei nimmt seine Wertigkeit ab.\\
	\hline
	\textbf{Oxidation}
		& Bei einer Oxidation nimmt die Elektronendichte um ein Atom ab, dieser Stoff wird \textbf{Reduktionsmittel} oder \textbf{Donator} genannt. Analog zur Reduktion nimmt seine Wertigkeit zu.\\
	\hline
	\textbf{Allgemeine Reaktionsschemata}
		& Eine Redoxreaktion lässt sich einfach in folgenden Reatkionsschemata verallgemeinern.
\end{longtable}
\begin{figure}[H]\centering
	\begin{subfigure}[H]{0.5\linewidth}\centering
		\chemfig{A} \chemrel{->} \chemfig{A^{+}} + \chemfig{e^{-}}
		\subcaption{Oxidation: Stoff \chemfig{A} (Donator) gibt ein Elektron ab}
	\end{subfigure}
	
	\begin{subfigure}[H]{0.5\linewidth}\centering
		\chemfig{B} + \chemfig{e^{-}} \chemrel{->} \chemfig{B^{-}}  
		\subcaption{Reduktion: Stoff \chemfig{B} (Akzeptor) nimmt ein Elektron auf}
	\end{subfigure}
	
	\begin{subfigure}[H]{0.5\linewidth}\centering
		\chemfig{A} + \chemfig{B} \chemrel{->} \chemfig{A^{+}} + \chemfig{B^{-}} 
		\subcaption{Redoxreaktion: Donator \chemfig{A} gibt ein Elektron an Akzeptor \chemfig{B} ab}
	\end{subfigure}
	\caption{Allgemeine Reaktionsschemata der Redoxreaktion}
\end{figure}

\subsection{Korrosion}
\begin{longtable}{p{3cm}p{14cm}}
	\hline
	\textbf{Definition}
		& Korrosion ist die Reaktion eines Werkstoffes mit seiner Umgebung, welche eine messbare Veränderung des Werkstoffes verursacht und zu einer Beeinträchtigung der Funktion des Bauteils oder des Systems führen kann. Korrosion tritt an Metallen auf, der Begriff ist jedoch auch in anderen Gebieten gebräuchlich.\\
	\hline
	\textbf{Voraussetzung für Korrosionsreaktionen}
		& Falls ein metallischer Stoff (ausser Gold und Platinmetalle) an der Luft beständig ist, muss er von einem Passivoxidfilm bedeckt sein, um eine Korrosion zu ermöglichen muss dieser zuerst zerstört werden. Korrosionen verlaufen demnach in zwei Teilschritten.
		\begin{enumerate}
			\item Depassivierung
			
			\item Korrosion (Redoxreaktion)
		\end{enumerate}
		Ob eine Depassivierung möglich ist wird durch den Gehalt an Depassivatoren und Passivatoren in der Elektrolytlösung bestimmt. Depassivatoren sind Stoffe, welche den Passivoxidfilm schwächen, Passivatoren sind Stoffe, die diesen stärken.
		Die Korrosionstypen werden zudem noch unterteilt in zwei Kategorien.
		\begin{itemize}
			\item \chemfig{H_{2}}-Typ: Oxidationsmittel \chemfig{H^{+}}
			
			\item \chemfig{O_{2}}-Typ: Oxidationsmittel \chemfig{O_{2}}
		\end{itemize}\\
\end{longtable}

\subsection{Das Eloxal-Verfahren}
\begin{longtable}{p{3cm}p{14cm}}
	\textbf{Definition}
		& Eloxal steht für \textbf{El}ektrolytische \textbf{Ox}idation von \textbf{Al}uminium. Dieses Verfahren ist eine Methode zur Erzeugung einer oxidischen Schutzschicht auf der Aluminiumoberfläche durch anodische Oxidation. Im Gegenzug zu galvanischen Verfahren wird beim Eloxal-Verfahren kein Überzug auf der Oberfläche gebildet, sondern die oberste Metallschicht oxidiert, so dass diese tiefer liegende Schichten vor Korrosion geschützt sind, so lange keine Schäden in dieser Schicht entstehen.\\
	\hline
	\textbf{Das Korrosionsverhalten von Aluminium}
		& Obwohl Aluminium ein unedles Metall ist, wird es durch eine sehr dünne, \chemfig{O_{2}}-undurchlässige atmosphärische Passivoxidschicht vor Korrosion geschützt. Die natürliche, atmosphärische, Oxidschicht von Aluminium ist nach einigen Tagen bis Wochen lediglich 100-500$\mathrm{nm}$ dick, wobei diese Eloxalschicht 5-25$\mathrm{\mu m}$ dick ist. Wird diese Oxidschicht verletzt, reagiert das Aluminium fast augenblicklich mit dem Sauerstoff zu Aluminiumoxid.\\
\end{longtable}

\begin{figure}[H]\centering
	$\chemfig{2Al} + \frac{3}{2}\chemfig{O_{2}} + x\chemfig{H_{2}O} \chemrel{->} \chemfig{Al_{2}O_{3}} + x\chemfig{H_{2}O} \quad x\in \{0, 1, 2, 3\}$
	\caption{Korrosionsverhalten von Aluminium}
\end{figure}

\begin{longtable}{p{3cm}p{14cm}}
	\hline
	\textbf{Vorbehandlung des Werkstücks}
		& Bevor ein Werkstück eloxiert wird muss dessen Oberfläche einen möglichst hohen Grad an Reinheit haben, da das Aussehen abgesehen von der Farbänderung nicht wesentlich verändert wird und demnach Beschädigungen danach gut sichtbar sind. Dementsprechend muss es  vorbehandelt werden. Dies beinhaltet in der Praxis Anschleifen, Polieren, Anätzen etc. Beim Anätzen wird das Werkstück in eine \chemfig{NaOH}-Lösung getaucht und in einem ersten Schritt die natürliche Passivoxidschicht entfernt (Depassivierung), in einem zweiten die Oberfläche wieder oxidiert.\\
\end{longtable}

\begin{figure}[H]\centering
	\begin{subfigure}[H]{0.5\linewidth}
		$$\chemfig{Al_{2}O_{3}}_{(s)} + 2\chemfig{OH^{-}} + 3\chemfig{H_{2}O} \chemrel{->} 2[\chemfig{Al}(\chemfig{OH_{4}})]^{-}_{(aq)}$$
		\caption{Depassivierung} 
	\end{subfigure}
	
	\begin{subfigure}[H]{0.5\linewidth}
		$$\chemfig{Al} + \chemfig{Na^{+}} + \chemfig{OH^{-}} + 3\chemfig{H_{2}O}\chemrel{->} \chemfig{Na^{+}} + \chemfig{Al}(\chemfig{OH})^{-}_4 + \frac{3}{2}\chemfig{H_{2}}$$
		\caption{Oxidation} 
	\end{subfigure}
	\caption{Beizen in der NaOH-Lösung}
\end{figure}

\begin{longtable}{p{3cm}p{14cm}}
	\hline
	\textbf{Verstärkung des Passivoxidfilmes}
		& Für die Herstellung dieser angebrachten dickeren, härteren und korrosionsbeständigeren Oxidschicht wird das Werkstück als Anode in der Elektrolyse mit einer Schwefelsäurelösung als Elektrolyt geschaltet. Um eine Verdickung der Passivoxidschicht zu erreichen, muss gleichzeitig die neugebildete Oxidschicht wieder aufgelöst werden. Dafür wird eine Säure benötigt, da die Schicht in neutraler Lösung nicht löslich ist. Ohne diese Depassivierung wäre nur eine Zersetzung vom Wasser möglich. 
		
		Während in dieser Elektrolyse an der Kathode \chemfig{H_{2}}-Gas gebildet wird, wird an der Anode Aluminium oxidiert, wobei sich eine fest haftende \chemfig{Al}(\chemfig{OH})\chemfig{_{3}}-Schicht aus den \chemfig{Al^{3+}}- und den \chemfig{OH^{-}}-Ionen bildet. Die \chemfig{OH^{-}}-Ionen entstehen durch die Autodissoziation des Wassers.\\
\end{longtable}

\begin{figure}[H]\centering
	\begin{subfigure}[H]{0.5\linewidth}
		$$6\chemfig{H_{3}O^{+}} + 6\chemfig{e^{-}} \chemrel{->} 6\chemfig{H_{2}O} + 3\chemfig{H_{2}}$$
		\caption{Reduktion an der Kathode} 
	\end{subfigure}
	
	\begin{subfigure}[H]{0.5\linewidth}
		$$2\chemfig{Al} \chemrel{->} 2\chemfig{Al^{3+}} + 6\chemfig{e^{-}}$$
		\caption{Oxidation an der Anode} 
	\end{subfigure}
	
	\begin{subfigure}[H]{0.5\linewidth}
		$$2\chemfig{Al} + 6\chemfig{H_{3}O^{+}} \chemrel{->} 2\chemfig{Al^{3+}} + 6\chemfig{H_{2}O} + 3\chemfig{H_{2}}$$
		\caption{Redoxreaktion} 
	\end{subfigure}	
	\caption{Bildung der Oxidschicht beim Eloxieren}
\end{figure}

\begin{longtable}{p{3cm}p{14cm}}
	\hline
	\textbf{Punktförmige Anlösung des Werkstoffes}
		& Nach wenigen Sekunden ist die Grundschicht auf ca 0.015$\mathrm{\mu m}$ angewachsen. Eine weitere Dickezunahme der Grundschicht ist danach aufgrund kleiner werdender Ionenleitfähigkeit nicht mehr möglich. Danach wird die Grundschicht von oxidlösenden Säureelektrolyten punktförmig angelöst. Diese Porenschicht wächst zu etwa $\frac{2}{3}$ nach innen und wegen des grösseren Volumens des Hydroxids zu etwa $\frac{1}{3}$ nach aussen. Dafür muss die Säurekonzentration so gewählt werden, dass die Schichtbildungsgeschwindigkeit grösser ist als die Schichtauflösungsgeschwindigkeit. Die dadurch maximal erreichbare Dicke ist ca 45 $\mathrm{\mu m}$, dann wird die Bildungs- und Auflösungsgeschwindigkeit etwa gleich gross.\\
	\hline
	\textbf{Einfärben der unverdichteten Eloxalschicht}
		& Da die Porenschicht nach dem Eloxieren saugfähig ist, können darin geeignete Farbstoffe eingelagert werden. Sie sollten möglichst tief in die Poren eindringen und darin festgehalten werden, Stellen, die nicht eloxiert wurden, werden nicht eingefärbt.\\
	\hline
	\textbf{Elektrolytisches Färben}
		& Beim elektrolytischen Färben wird das zu färbende Aluminium und eine Graphitelektrode an eine Wechselstromquelle angeschlossen. Die Elektroden werden in einen Elektrolyten getaucht, welcher farbige Metallsalze enthält. Während der einen Periodenhälfte dringen die Metallionen in die Poren ein und entladen sich an deren Grund, in der zweiten Hälfte wird die Sperrschicht verstärkt und stabilisiert.\\
	\hline
	\textbf{Verdichten}
		& Um die Poren dauerhaft zu schliessen wird die Oxidschicht normalerweise verdichtet. Dies geschieht durch die Volumenzunahme der Eloxalschicht durch die Erwärmung im siedenden Wasser. Das Aluminiumoxid reagiert zu wasserreichem $\alpha-\chemfig{Al}(\chemfig{OH})_3$, welches durch die steigenden Temperaturen langsam zu kristallinem Böhmit($\gamma-\chemfig{AlO}(\chemfig{OH})$) reagiert. Dank dessen grösseren Volumen dehnt sich die Schicht aus und schliesst somit die Poren.\\
	\hline
\end{longtable}

\subsection{Galvanisches Verkupfern, Vernickeln und Vergolden}

\begin{longtable}{p{3cm}p{14cm}}
	\textbf{Definition von Galvanisieren}
		& Unter Galvanisieren, auch Elektroplattieren genannt, versteht man die elektrochemische Abscheidung von metallischen Überzügen auf Gegenständen. Dabei muss der Gegenstand als Kathode in der Elektrolysezelle geschaltet werden und das Überzugsmaterial als Kation in der Lösung vorliegen.\\
	\hline
	\textbf{Vorbehandlung des Werkstücks}
		& Auch bei der Galvanisierung muss das Werkstück für qualitativ hochwertige und optisch ansprechende Überzüge vorbehandelt werden.
		\begin{itemize}
			\item \textbf{Grobreinigung}
			
				Grobe Verunreinigungen und Korrosionsprodukte werden mit Hilfe von Putzmittelpaste, warmem Wasser und Ultraschall entfernt.
			
			\item \textbf{Elektrolytische Entfettung}
			
				Die Metallprobe wird in eine stark basische Elektrolytlösung eingetaucht und darin als Kathode. Die Anoden bestehen aus Chromnickelstahlplatten. Durch die \chemfig{H_{2}}-Bildung an der Kathode und den zusätzlich beigegebenen Emulgatoren werden die Fettmoleküle auf der Oberfläche des Werkstoffes von den Gasblasen weggespült.
				
			\item \textbf{Oxidfilmentfernung}
			
				In einem sauren Bad mit Netzmittelzusatz wird der Oxidfilm entfernt und gleichzeitig die eventuell noch basische Metalloberfläche neutralisiert. Dadurch wird eine sofortige, gleichmässige Benetzung beim Eintauchen ins Galvanisierbad gewährleistet.
		\end{itemize}\\
	\hline
	\textbf{Verkupfern}
		& \textcolor{red}{\textbf{Achtung: Die Elektrolytlösung darf unter keinen Umständen sauer werden!! Sonst entsteht aus dem Zyanid hochtoxische Blausäure, welche als Gas aus der Lösung entweicht!!}}
	\end{longtable}
	\begin{figure}[H]\centering
		$$\chemfig{CN^{-}} + \chemfig{H_{3}O^{+}} \chemrel{->} \chemfig{HCN} + \chemfig {H_{2}O}$$
		\caption{Bildung von Cyanid in der sauren\\ Elektrolytlösung beim Verkupfern}
	\end{figure}
	
\begin{longtable}{p{3cm}p{14cm}}
	\hline
	\textbf{Zerfall des Hauptelektrolyten}
		& Der Hauptelektrolyt $\chemfig{Na_{2}Cu}(\chemfig{CN})_{3}$ (Natriumkupfercyanid) zerfällt beim Verkupfern in zwei Bestandteile:
\end{longtable}
\begin{figure}[H]\centering
	$$\chemfig{Na_{2}Cu}(\chemfig{CN})_{3} \chemrel{->} 2\chemfig{Na^{+}}_{(aq)} + \chemfig{Cu}(\chemfig{CN})^{2-}_3$$
	\caption{Zerfall des Hauptelektrolyten\\ beim Verkupfern.}
\end{figure}

\begin{longtable}{p{3cm}p{14cm}}
		& Darin stellt sich folgendes Gleichgewicht ein:
\end{longtable}

\begin{figure}[H]\centering
	$$\chemfig{Cu}(\chemfig{CN})^{2-}_3 \chemrel{<->} \chemfig{Cu^{+}} + 3\chemfig{CN^{-}} \quad K = 10^{-20.8}$$
	\caption{Chemisches Gleichgewicht im\\ Hauptelektrolyten beim Verkupfern.}
\end{figure}

\begin{longtable}{p{3cm}p{14cm}}
		& wobei die Konzentration der Produkte wesentlich kleiner ist als die Konzentration der Edukte, erkennt man, dass bei einer angelegten Spannung die \chemfig{Cu^{+}} zum, als Kathode geschalteten, Werkstück gezogen werden.\\
	\hline
	\textbf{Erklärung mit dem Le Châtelier-Gesetz}
		& Durch das Gesetz des kleinsten Zwanges von Le Châtelier kann man die Qualitative Aussage machen, dass das chemische Gleichgewicht aufrecht erhalten werden will, und dem entsprechend von der positiv geladenen Kupferanode weitere Kupferkationen herauslöst.
		
		Daraus ergeben sich folgende Reaktionsgleichungen:\\
\end{longtable}

\begin{figure}[H]\centering
	\begin{subfigure}[H]{0.5\linewidth}
		$$\overset{0}{\chemfig{Cu}_{(s)}} \chemrel{->} \overset{-\mathrm{I}}{\chemfig{Cu^{+}}_{(aq)}} + \chemfig{e^{-}}$$
		\caption{Oxidation von Cu an der Anode} 
	\end{subfigure}
	
	\begin{subfigure}[H]{0.5\linewidth}
		$$\overset{-\mathrm{I}}{\chemfig{Cu^{+}}_{(aq)}} +\chemfig{e^{-}} \chemrel{->} \overset{0}{\chemfig{Cu}_{(s)}}$$
		\caption{Reduktion von Cu an der Kathode} 
	\end{subfigure}
	
	\caption{Reaktionsgleichungen an der\\ Anode und Kathode beim Verkupfern}
\end{figure}

\begin{longtable}{p{3cm}p{14cm}}
\hline
\textbf{Vernickeln}
	& Beim Vernickeln zerfällt der Hauptelektrolyt \chemfig{NiSO_{4}} (Nickelsulfat) folgendermassen:
\end{longtable}

\begin{figure}[H]\centering
	$$\chemfig{NiSO_{4}} \chemrel{->} \chemfig{Ni^{2+}} + \chemfig{SO_{4}^{2-}}$$
	\caption{Zerfall des Hauptelektrolyten\\ beim Vernickeln.}
\end{figure}

\begin{longtable}{p{3cm}p{14cm}}
\hline
	& Diese \chemfig{Ni^{2+}}-Ionen werden hydratisiert, diese hydratisierten Nickelionen reagieren mit Wasser schwach sauer:
\end{longtable}

\begin{figure}[H]\centering
	$$\chemfig{Ni}(\chemfig{H_{2}O})^{2+}_6 + \chemfig{H_{2}O} \chemrel{->} [\chemfig{Ni}(\chemfig{H_{2}O})_5\chemfig{OH}]^{+} + \chemfig{H_{3}O^{+}} \quad K_s = 10^{-9.9}$$
	\caption{Schwach saure Reaktion der\\hydratisierten Nickelionen bei \\ der Vernickelung}
\end{figure}

\begin{longtable}{p{3cm}p{14cm}}
\
\end{longtable}